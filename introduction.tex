% !TEX root = main.tex
\section{Introduction}

%Semantifying the hyperlinks in web documents is a huge task.
Recent years have witnessed the booming of semi-structured documents, especially online encyclopedia such as  Wikipedia~\footnote{\small\url{http://www.wikipedia.org}}, which serves important roles in offering knowledge services.
{\it Wikipedia.org} is consistently among the top 10 most visited websites according to Alexa Traffic Ranks~\footnote{\small\url{http://www.alexa.com/siteinfo/http://www.wikipedia.org}}.
Wikipedia is flooded with hyperlinks, which interconnect the Wikipedia articles.
In general, one article contains a hyperlink to another article because the entities corresponding to the articles have a certain semantic relationship.
However, explanation of such relationship is a non-trivial problem.

\subsection{Problem Statement}
We thus study this problem of{\it semantifying the hyperlinks in Wikipedia}, of 
explaining the relationship between an entity pair whose corresponding Wikipedia articles are hyperlinked with each other.
For example, Wikipedia page of \at{Leonardo Da Vinci}, has a hyperlink to \at{Mona Lisa} the sentence \at{... 
Among his works, the \underline{Mona Lisa} is the most famous and most parodied portrait}.
Our goal is, for each Wikipedia page, representing an entity or a concept,
we find an \emph{attribute}, such as \emph{ArtistOf}, which best explains its relationship with a hyperlinked entity.


\subsection{Our Core Idea}

This problem is trivial, if DBpedia contains 
SPO (subject-predicate-object) triples \at{<Leonardo Da Vinci, ArtistOf, Mona Lisa>} which indicate the predicate \at{ArtistOf} (which is an attribute of \at{Leonardo Da Vinci}) can explain the relation between \at{Leonardo Da Vinci} and \at{Mona Lisa}. However, this case is extremely rare:
Our statistics show that there are \emph{968 millions} of unique links between Wikipedia entities, only \emph{15 millions} (1.5\%) of which can find semantic explanation from DBpedia.

Thus, {\it how to improve the recall when the relationship facts are not covered by the knowledge base?}
A key observation is that the concept pairs
that best explain the given entity pairs can be effective intermediate 
variables. 
For example, \at{<Leonardo Da Vinci, Mona Lisa>} can be best explained by \at{ArtistOf} attribute, 
which typically holds between concept pair \at{<artist, painting>}. 
Many SPO triples \at{<Leonardo Da Vinci, ArtistOf, Mona Lisa>}, \at{<Salvador Dali, ArtistOf, The Persistence of Memory>}
enable the conclusion that \emph{ArtistOf} is a typical attribute for \at{<artist, painting>} concept pair.

This enables the understanding of entity pairs unobserved in DBpedia triples, if they can be correctly conceptualized into \at{<artist, painting>}.
Though conceptualizing a word or phrase has been actively studied~\cite{song2011short,kim2013context}, conceptualization in our problem setting has the following unique characteristics:


\subsection{Challenges and Contributions}

\begin{itemize}
\item First,  {\it joint conceptualization.} Conceptualization of
$e_1$ and $e_2$ is not independent, and thus requires joint optimization, 
as we illustrate in Example~\ref{exa:jc}.
\item Second, {\it collective conceptualization.} 
Conceptualization of a single entity pair is more ambiguous than that of many pairs, for which we \emph{collectively} conceptualize for a set of pairs observed from SPO triples.
We illustrate this in Example~\ref{exa:sd}.
\item Third, {\it head-aware conceptualization}. As illustrated in Example~\ref{exa:hc}, in our problem setting, it is the head concept determines the relationship between an entity pair. Hence, we need to conceptualize entity into its head concepts instead of some other modified concepts. 
\end{itemize}

{\footnotesize

\begin{example}[Joint conceptualization]
Given an entity pair \at{<apple, steve jobs>}. 
The candidate concepts for \at{apple} are \at{\{fruit,company,...\}} and the candidate concepts for \at{steve jobs} are \at{\{entrepreneur, pc developer,...\}}. 
Obviously, the concept pair \at{<fruit, entrepreneur>} is meaningless in this case if we consider the concept pair jointly.
\label{exa:jc}
\end{example}


\begin{example}[Collective conceptualization]
Continue the previous example, when we are given a set of relations from SPO triples, such as \at{<microsoft, bill gates>}, \at{<ibm, thomas watson>}, who share the same attribute \at{FoundBy} with \at{<apple, steve jobs>}, we can 
avoid conceptualizing the pair into
\at{<fruit, entrepreneur>}.
\label{exa:sd}
\end{example}

\begin{example}[Head-aware conceptualization]
The \at{FoundedBy} relationship between \at{Apple Inc.} and \at{Steve Jobs} is determined by the head concepts they possessed (e.g.\ \term{company} and \term{entrepreneur}, while this observation may be diluted into \emph{long concepts} with  modifiers such as \term{technology company} or \term{successful entrepreneur}.)
\label{exa:hc}
\end{example}

}

%Challenges are as follows:
%\begin{description}
%  \item[C1] Entity Representation.
%  Finding the most plausible concept pairs for the entity pair is hard due to the sparsity of concepts.
%  The average number of parent concept~\cite{wu2012probase} for an entity is 1.04 in Yago and 2.33 in Probase.
%  Furthermore, an complete representation need to include probability information to reflect typicality rather than black-or-white assertion of the type.
%  \item[C2] Large search space.
%  Finding the most plausible relation for the concept pair distribution is time consuming due to the large concept pair space.
%  The exact inference requires to find maximal value from a C-C space that has over 4 trillion concepts' combination(probase has 2 million concepts )
%\end{description}



%We argue that the context of an entity in an article is informative[] and deserve a higher occurrence[] and can produce fresh and latest relation of an entity, which can be useful in updating the KB[reverb].

%In our approach the relations is not necessarily to be exist, however the cluster of the relation[relation clustering] it belongs to should be conceptually right. With rich context and large knowledge base, we can easily derive the fresh context relations and the concept of each entity,

%On the other hand, we consider the co-occurrence of the 2 entities, based on the assumption of important relationship will be observed in various of documents[]

We propose a probabilistic framework, integrating probabilities observed
from Wikipedia and taxonomies.
We validate our framework using real-life Wikipedia entity pairs and Probase~\cite{wu2012probase} taxonomy.

