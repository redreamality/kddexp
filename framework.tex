\section{Framework}


In this section, we present the problem and our solution to it.

\subsection{Problem Statement}

For a given pair of entity, we formalize the problem of explaining the relationship between into getting a ranking of all the possible attributes. Our target is to calculate the typicality of an attribute that can best define the possible relation of the 2 entities, which can be denoted as $P(a| (e_1,e_2) )$. One possible way is to store all the entities and their relationships, so that we can find an accurate attribute for each entity pair. Unfortunately, it is almost impossible to achieve this since there are billions of entities and new entities emerges everyday. Thus we need to utilize the concepts of the entities. Eq.~\ref{eq:target} breaks down our target into 2 parts: 1)Calculating the typicality of an attribute towards a concept pair. 2) Calculating the typicality of the concept pair towards an entity pair.
\begin{equation}
\label{eq:target}
P(a| (e_1,e_2) )= \sum_{c_1\in C_1 ,c_2 \in C_2 }P(a|(c_{1},c_{2}))\times P((c_{1},c_{2})|(e_{1},e_{2})),
\end{equation}

where $P((c_{1},c_{2})|a)$ represents the probability of given an attribute, how likely is a concept pair going to appear. 
We use Bayesian theorem to convert the first part in Eq.\ref{eq:target_expand1}:

\begin{equation}
\label{eq:target_expand1}
\begin{split}
P(a|(c_{1},c_{2})) &= \frac{ p((c_{1},c_{2})|a)\times P(a) }{ P( (c_{1},c_{2}) ) }\\
&=\frac{ p((c_{1},c_{2})|a)\times P(a) }{ \sum{P( (c_{1},c_{2})|a^* )\times P(a^*)   } },
\end{split}
\end{equation}

where $P(a)$ is defined by Eq.~\ref{eq:pa}, denoting how frequent attribute occurs.
\begin{equation}
\label{eq:pa}
P(a)=\frac{n(a)}{\sum{n(a^*)}},
\end{equation}
where $n(a)$ is the count of the attributes.

$P((c_{1},c_{2})|(e_{1},e_{2}))$ denotes the likelyhood of the occurrence a concept conbination given the entity, one naive solution is as Eq.~\ref{eq:target_expand2_naive}, following the intuition of finding typical concepts for the entity. We discuss how to calculate $P(c|e)$ for this task in detail in Section.~\ref{sec:conceptualization}.

\begin{equation}
\label{eq:target_expand2_naive}
\begin{split}
P((c_{1},c_{2})|(e_{1},e_{2})) = P(c_1|e_1) \times P(c_2|e_2)
\end{split}
\end{equation}

However, the joint distribution of the two concepts is indispensable for sense disambiguation. For example, we suppose the given entity pair is (apple, steve jobs). The potential left concepts are \term{fruit,company,...} and the right concepts are \term{entrepreneur, pc developer,...}. Obviously, we cannot combine \term{fruit} with \term{entrepreneur} in this case. Therefore, the Joint Distribution $JD(c_1,c_2)$ is introduced in Eq.~\ref{eq:target_expand2_jr}

\begin{equation}
\label{eq:target_expand2_jr}
\begin{split}
P((c_{1},c_{2})|(e_{1},e_{2})) = JD(c_1,c_2) \times P(c_1|e_1) \times P(c_2|e_2)
\end{split}
\end{equation}


Combining all these together, the complete version of $P(a|(c_{1},c_{2}))$ is described in Eq.~\ref{eq:target_expand_all}

\begin{equation}
\label{eq:target_expand_all}
\begin{split}
 P(a| (e_1,e_2) ) &= \frac{ p((c_{1},c_{2})|a)\times P(a) }{ \sum{P( (c_{1},c_{2})|a^* )\times P(a^*)   } }\\
 &\times JD(c_1,c_2) \times P(c_1|e_1) \times P(c_2|e_2)
\end{split}
\end{equation}



\subsection{Problem Solution}
Thus, the main framework is divided into 2 parts, the online part and the offline part.
\paragraph{The offline part}

According to the above derivation, there are 2 aspect we need to calculate offline: $JD(c_1,c_2)$ and $P((c_{1},c_{2})|a)$. 
We leverage the plentiful $(e_1,a,e_2)$ tuple in DBpedia and their concepts to compute




\paragraph{The online part}
For an entity pair request, we can calculate $ P(a| (e_1,e_2) )$ using Eq.~\ref{eq:target_expand_all}, where we get a typicality score for each attribute, once we get $P((c_{1},c_{2})|a)$ and $JD(c_1,c_2)$. Hence, we get a ranking of the attributes for the entity pair. 


\paragraph{Paper Orgnization}
The rest of the paper is organized as follows, Section~\ref{sec:conceptualization} describe how to derive $P(c|e)$ leveraging the \xch{basicness} of concept, Section~\ref{sec:fafa} is devoted to the offline calculation of $JD(c_1,c_2)$ and $P((c_{1},c_{2})|a)$. 

%First,we do conceptualization.
%
%Next,Judge whether the 2 entities are conceptually same
%
%Then, there are 2 cases of the CanBeExplained function:
%
%\begin{itemize}
%\item Explain 2 conceptually similar entity
%
%\begin{table}[htbp]
%  \centering
%  \caption{conceptually similar entity}
%    \begin{tabular}{rr}
%    \toprule
%    entity & concept \\
%    \midrule
%    Steve jobs & Person \\
%    Bill Gates & Person \\
%    \bottomrule
%    \end{tabular}%
%  \label{tab:addlabel}%
%\end{table}%
%
%
%\item Explain 2 conceptually different entity
%% Table generated by Excel2LaTeX from sheet 'Sheet1'
%\begin{table}[htbp]
%  \centering
%  \caption{Add caption}
%    \begin{tabular}{rr}
%    \toprule
%    entity & concept \\
%    \midrule
%    Mona Lisa & Painting \\
%    Renaissance & Period \\
%    \bottomrule
%    \end{tabular}%
%  \label{tab:addlabel}%
%\end{table}%
%
%Note that the concept here are not unique.
%
%\end{itemize}
%
%
%Last, We rank all the explanations in each step.
%
%
%
