
% !TEX root = main.tex
\section{Problem Model}
\label{sec:framework}

In this section, we formalize our problem, followed by a restatement based on concept based inference.

\paragraph{Problem Formalization}
For a given pair of entity, we formalize the problem of explaining the relationship between them as finding the attribute with a probability ranking.
More formally, given an entity pair $ \langle e_1, e_2 \rangle $, we want to find:
\begin{equation}
\argmax_a P(a|\langle e_1,e_2 \rangle),
\end{equation}
where $P(a| \langle e_1, e_2 \rangle )$ is the probability that $a$ is an attribute between two entities. Similarly, we later use $P(a| \langle c_{1},c_{2} \rangle )$ to denote the {\it typicality} of attribute $a$ given the concept pair.

For example, for entity pair \at{<Bill Gates, Microsoft>}, \at{FounderOf, BoardOf} are among the most probable relations.
However relation like \at{KeyPeopleOf} is not that typical. 

%We give notations used in this paper in Table~\ref{tab:notation}.

\paragraph{Concept-based Inference}
Thus our problem is reduced to the estimation of $P(a| \langle e_1, e_2 \rangle )$.
The direct estimation is difficult since no such samples are available.
We resort to the concepts of entities to bridge the inference from entity pair to their attribute.
The rationality comes from the observation that {\it it is the concept determines the relationship between entities}. For example, \at{<Washington, USA>} can be best explained by \at{CapitalOf} attribute, which essentially is determined by the concept pair \at{<Capital City, Country>}.
In other words, a relationship can be considered as generated by corresponding concept pairs.
The entity pairs that belong to the same concept pairs share the same relationship explanation.
Continue the previous example, \at{<Beijing, China>} is clearly another example of entity pairs belonging to \at{<Capital, Country>} that can be explained by the \at{CapitalOf} attribute.

Using concept pairs as intermediate random variables, our problem can be restated as:
\begin{equation}
\label{eq:target}
\small
\argmax_a \sum_{c_i\in C_1 , c_j \in C_2 }P(a|\langle c_{i},c_{j}\rangle)\times P(\langle c_{i},c_{j}\rangle|\langle e_{1},e_{2}\rangle),
\end{equation}
where $P(a|\langle c_{1},c_{2}\rangle)$ is the probability that attribute $a$ is relationship between the concept pair
and $P(\langle c_{i},c_{j}\rangle |\langle e_{1},e_{2}\rangle)$ is the probability that $\langle c_1, c_2\rangle$ is the concept pair of $ \langle e_1, e_2 \rangle $.
%\nop{
%For the same concept pair, some attribute is more typical than another one.
%For concept pair \at{<artist, country>}, the attribute \ac{hometown} is more typical than \ac{education}.
%For a given entity pair, some concept pair is better to characterize the concept of the entity pair than others.
%As example,  for \at{<apple, Steve Jobs>},  \at{<company, entrepreneur>} deserves a higher probability than \at{<food, name>}.
%$P( \langle c_{i},c_{j} \rangle | \langle e_{1},e_{2} \rangle )$ can also be considered as the typicality of the concept pair for the given entity pair.
%
%
%Another benefit of concept-level estimation is that it can reduce the computation cost.
%In general, the number of concept pairs is significantly less than that of entity pairs.
%}
