\section{Related Works}
\label{sec:rel}
%improves: incrementally training the entity-concepts so that it can model time varying attributes.(phone,manufacturer,company)--after2007-->(smartphone ,manufacturer,company)
%perform NER~\cite in the document to link the entities.
%\subsection{Relation Explanation}

Various of efforts are devoted to relation explanation.
Blanco et al.~\cite{blanco2010finding} study the problem of finding support sentences to explain the relationship between a named entity and an ad-hoc query.
Fang et al.~\cite{fang2011rex} explain the relation ship between 2 entities by answering a subgraph from the knowledge base.
Voskarides et al.~\cite{voskarideslearning} models the problem of explaining relationship between entities into sentences retrieval problem, ranking the explanation sentence from Wikipedia by how well it describe the relationship.
All these tasks takes the relation between entities as given input from knowledge graph, in this paper, we retrieve the most probable relation for a given entity pair.

Relation extraction is a well-studied task.
Many efforts are devoted to extracting relations from sentence.
Early work~\cite{hearst1992automatic,brin1999extracting,agichtein2000snowball} and some wiki-based~\cite{ponzetto2008wikitaxonomy} approaches applies hand-crafted patterns.
supervised classification~\cite{doddington2004automatic,guodong2005exploring} and distant learning~\cite{craven1999constructing,wu2007autonomously,bunescu2007learning,mintz2009distant}.
Recent effort~\cite{angeli2014combining} applied active learning and combines distant and partial supervision.
These efforts, including NELL are based on  fixed set of prespecified relations.
Another type of relation extraction is open relation extraction~\cite{banko2007open}, such as Reverb~\cite{fader2011identifying}, OLLIE~\cite{schmitz2012open} are unsupervised approaches extracting relations from plain text.
PATTY~\cite{nakashole2012patty} also extract open relational patterns from wikipedia and organize them into a taxonomy.
Our main difference from all of the approaches is that, our input is entity pair instead of sentences.
Our relation set is also fixed, specifically the ones in DBpedia~\cite{auer2007dbpedia}.
Our work is closely related to generative models, instead of the latent variables used in Rel-LDA model in~\cite{yao2011structured}, we use rich concepts to represent entities, though our tasks are different.



%\subsection{Link Label Prediction}

%Link Label Prediction~\cite{agrawal2013link}

%\subsection{Information Extraction}
%
%Attribute acquisition methods
%Among domain-dependent approaches, we can mention approaches that focus on products. In this domain, attributes have been used to improve product search and recommendation [18, 22], but also to enable data mining [27]
%
%Attribute retrieval provides another granularity in Web
%search. This can interest communities that propose a more
%focused access to information or communities that envision
%aggregating pieces of information such as aggregated search
%[19, 15].


%\subsection{Link entities to Database}
%
%Linking entities to database, especially to Wikipedia, has been widely studied. Entity linking to Wikipedia~\cite{milne2008learning,mihalcea2007wikify,han2011collective} exploit Wikipedia as thesaurus and link web documents to it.
%In our work, instead of linking entities to the correspond one in KB, we extract the target entity~\cite{dalvi2011automatic} and explain the semantic relation of the entity towards target entity.


%\subsection{Short Text Conceptualization}
