\section{Related Works}
%improves: incrementally training the entity-concepts so that it can model time varying attributes.(phone,manufacturer,company)--after2007-->(smartphone ,manufacturer,company)




%perform NER~\cite in the document to link the entities.

\subsection{Relation Explanation}


Various of efforts are devoted to relation explanation.
Blanco et al.~\cite{blanco2010finding} study the problem of finding support sentences to explain the relationship between a named entity and an ad-hoc query.
Fang et al.~\cite{fang2011rex} explain the relation ship between 2 entities by answering a subgraph from the knowledge base.
Voskarides et al.~\cite{voskarideslearning} models the problem of explaining relationship between entities into sentences retrieval problem, ranking the explanation sentence from Wikipedia by how well it describe the relationship.
All these tasks takes the relation between entities as given input from knowledge graph, in this paper, we retrieve the most probable relation for a given entity pair.

Relation extraction is a well-studied task.
These approaches 
~\cite{shahaf2010connecting,luo2007answering} are on graph based approaches and text based approaches~\cite{hasegawa2004discovering}[reverb].
\xch{~\cite{nakashole2013discovering,nakashole2012discovering,konstantinova2014review}}
%However, these approached will be limited largely by the incompletion of Knowledge Base, and cannot discover new of relations.
In our work, we focus on semantic relations of entity by leveraging the concepts of entity. extracted from knowledge bases to link entities with a probability.


\subsection{Link Label Prediction}

Link Label Prediction~\cite{agrawal2013link} 

\subsection{Information Extraction}

Attribute acquisition methods
Among domain-dependent approaches, we can mention approaches that focus on products. In this domain, attributes have been used to improve product search and recommendation [18, 22], but also to enable data mining [27]

Attribute retrieval provides another granularity in Web
search. This can interest communities that propose a more
focused access to information or communities that envision
aggregating pieces of information such as aggregated search
[19, 15].
Wong et al. [27]
combine tags and textual features in a Conditional Random
Fields model to learn attribute extraction rules, but they
need a seed of relevant documents manually fed.


\subsection{Link entities to Database}

Linking entities to database, especially to Wikipedia, has been widely studied. Entity linking to Wikipedia~\cite{milne2008learning,mihalcea2007wikify,han2011collective} exploit Wikipedia as thesaurus and link web documents to it.
In our work, instead of linking entities to the correspond one in KB, we extract the target entity~\cite{dalvi2011automatic} and explain the semantic relation of the entity towards target entity.


\subsection{Short Text Conceptualization}
