\section{Probability Recalculation After Conceptualization}
\label{sec:conceptualization}


\subsection{Problem Statement}

\xch{intro about probase}

Given an entity $e$, from Probase, we can acquire its concepts' set $C$ and for each $c_i \in C$, the frequency $n(c_i,e)$ can be accordingly derived, which means how many times the $e$ isA $c_i$ pattern can be observed from the original corpus.we can derive $P(c_i|e)$, where $c_i \in C_{probase}$,
$$P(c_i|e)=\frac{n(c_i,e)}{n(e)}$$

However, the concepts here has various forms as illustrated in Example~\ref{exa:conc}. For our task, we only need relatively general concepts(a.k.a. head concepts).
\begin{example}[Various forms of concepts]
\label{exa:conc}
Take the entity \term{Mona Lisa} as example, its concepts includes \term{painting, famous painting, world's most famous painting},  with corresponding frequency \term{33,8,1}
\end{example}

We divide the $C_{Probase}$ into 2 parts, $C_{simple}$ and $C_{l}$, where $C_{simple}$ only contains one word and $C_{l}$ are the rest.
$C_{simple}$ are generated by the head modifier detection. The problem here is to recalculate the probability $P({c_h}|e)$ where ${c_h} \in C_{simple}$, literally, we should contribute all the counts of $C_{l}$ to $C_{simple}$.


\paragraph{Why head concepts?}
The relationship between entities are determined by the simple concepts. For example, the \term{founder} relationship between \term{Apple Inc.} and \term{Steve Jobs} are determined by the head concepts they possessed (e.g.\ \term{company} and \term{entrepreneur}, regardless of the modifiers such as \term{technology} in the concept \term{technology company}.)
The number of entities can be very large, but the number of top concepts and the relationship between them are limited, literally, we can find all the possible relationship between concepts instead of store all the long-tailed entities and their relations, which indicates the rationality of doing conceptualization.
The reason why we do head modifier detection instead of directly using isA edge in probase, is that even if the long concept $c_{l}$ has an isA edge towards a certain concept ${c_h}'$, it still sometimes not include the head concept of the long concept which is very plausible as is demonstrated in Example~\ref{exa:HvsO}

\begin{example}[Head concepts VS Original concepts]
\label{exa:HvsO}
Take \term{famous painting} as example. Its original concepts are \term{image, treasure}, which are reasonable but not plausible, since their occurrence are 2 and 1 respectively. However, the most plausible concept \term{painting} is not among the concepts.
\end{example}

\paragraph{The main steps}
Given an entity $e$ from $Probase$, we can get its concepts from probase. First we do head modifier detection based on syntax[], since the concepts in Probase all follows English grammar, this approach already produces a good result. Next, we recalculate the probability of $P({c_h}|e)$ by aggregating the contribution from $c_l$. The essentiality of doing so is illustrated in Example~\ref{exa:recalc}.




\begin{example}[Essentiality of Aggregation]
\label{exa:recalc}
\term{steve jobs} The concept\term{well-known name} has four occurrences however \term{name} has only 2.
\end{example}



\subsection{Baseline}
After head modifier detection, we have a set of ${c_h} \in C_{simple}$, among all the $c_{l_j}\in C_{l}$, there are 2 cases in the probase determined by whether the $c_{l_j}$ has an isA edge towards ${c_h}$ or not.
The intuition of doing so is illustrated in the Example~\ref{exa:clc}:

\begin{example}[contributing long concepts]
\label{exa:clc}
Assume that  \term{Mona Lisa is a painting} and \term{Mona Lisa is a famous painting} are observed respectively \term{33 times and 8 times} from different documents, we will get the knowledge that \term{Mona Lisa is a painting} occurs \term{41 times} instead of \term{33 times}.
\end{example}

Hence, the most straight forward approach is to contribute the corresponding long concepts to the simple ones as follows:
$$\hat{P}(c_h|e)=\frac{n(c_h,e)+\sum_{c_h= f_{HM}(c_l*)} n(c_l*,e)}{ n(e)} $$
where $f_{HM}()$ is a function that takes a long concept and produce a head concept.

\subsection{Combined Model with Original IsA}

In this section, we take the original Probase IsA relation into consideration. In Example.~\ref{exa:isagood}, we

\begin{example}[Resonable IsA Relation]
\label{exa:isagood}
  There exists several original IsA concepts of the long concepts that are also reasonable. For example \term{ topaz}(a kind of yellow gemstone) has the concept \term{precious stones}, and \term{precious stones} has an edge towards \term{material} which is reasonable.
\end{example}

Therefore, to calculate  $P({c_h}|e)$, there are three cases:

\begin{itemize}

\item[Case A]  $e$ \isa ${c_h}$.
 The entity has has an isA edge towards one or more simple concept, which gives the original $P_{original}({c_h}|e)$

\item[Case B.1] $e$ \isa $c_{l}$  \isa ${c_h}$, In this case, we need to calculate the following equation
$$P({c_h}|e) = \sum_{c_{l}^*\in C_{l}}   P({c_h}|c_{l}^*,e)   \times    P(c_{l}^*|e) $$
, where $P(c_{l}^*|e)$ can be obtained from $Probase$ and
\begin{equation}P({c_h}|c_{l},e) = \frac{n({c_h},c_{l}, e)}{n({c_h}, e)}\label{eq:pcge}\end{equation}
We assume that the occurrence of $e$ does not affect $P({c_h}|c_{l})$ equivalently speaking, $P({c_h}|c_{l})$ is independent from $e$, thus Eq.~\ref{eq:pcge} can be simplified
$$P({c_h}|c_{l},e) =P_{probase}({c_h}|c_{l}) = \frac{n({c_h},c_{l})}{n({c_h})}$$
which can be obtained from $Probase$.

\item[Case B.2] $e$ \isa $c_{l}$ \noisa ${c_h}$. The solid edge here refers to the isA relationship in $Probase$ and the dashed one refers to the edge generated by head modifier detection. Example~\ref{exa:wahro} pointed out that there won't be necessarily an isA edge from \term{famous painting}($c_{l}$ ) to \term{painting}(${c_h}$), however $c_{l}$ is obviously a hyponym of ${c_h}$. In this case,
    % since it's detected by the head modifier method, we assume
    % $$P_{head}({c_h}|c_{l_j})=1 $$
    we have to re-calculate the $P({c_h}|{c_l})$.
    In the original probase approach, we use Eq.~\ref{phgl_org} to calculate the probability.
    \begin{equation}\label{phgl_org} P({c_h}|{c_l}) = \frac{n( {c_h},{c_l} )}{ \sum{n( {c_h}^*,{c_l} )}  } \end{equation}
    However, $n( {c_h},{c_l} )$ is lower than expected due to the reason demonstrated in Example.~\ref{exa:wahro}.
    Therefore, we alternatively utilize the $\sum{ e^*,n({c_l}) } $ as the occurrence of $c_l$, following the assumption that \em{ whether $c_l$ is typical towards its $c_h$ is independent from }
    $$\hat{P}(c_h|e)=\frac{n(c_h,e)+\sum_{c_h= f_{HM}(c_l*)} n(c_l*,e)}{ n(e)} $$




\end{itemize}

\begin{example}[Why aren't head relationship observed]\label{exa:wahro}
There are less chance of occurring \term{Famous painting is a painting} in the corpus, since human takes it for granted and will seldom express it in such a way, so that there won't be necessarily an isA edge from \term{famous painting} to \term{painting} in the KB, while we insist it is necessary.
\end{example}

In both case B.1 and B.2, the weight of the edge $c_{l}$ \isa ${c_h}$ is underestimated. We argue that when calculating the typicality $P({c_h}|e)$, the counts of the long concept contributing to its head concept should be re-estimated as follows.

Notice that the boundary between case B.1 and case B.2 are not strict, there are such edges that have low observation in Example~\ref{exa:HvsO}. So that if we consider them as a whole, we can derive:
\begin{equation} P({c_h}|c_{l})=\lambda P_{head}({c_h}|c_{l})+(1-\lambda)P_{probase}({c_h}|c_{l}) \label{eq:pcgclong}\end{equation}
where $\lambda$ is a parameter \xch{principle: related to plausibility, number of occurrence, varies for different $c_{l}$ should it be derived from learning ?} since we assume $P_{head}({c_h}|c_{l})$ to be 1, Eq.~\ref{eq:pcgclong} is simplified to:
$$P({c_h}|c_{l})=\lambda  +(1-\lambda)P_{probase}({c_h}|c_{l}) $$






Finally  $P({c_h}|e)$ is calculated using the following equation:

\begin{equation}
\begin{split}
P({c_h}|e) &= P_{original}({c_h}|e)+\\& \sum_{ c_{l}^*\in C_{l} } [ \lambda_{i}^*+(1-\lambda_{i}^*) P({c_h}|c_{l}^*) ] \times  P(c_{l}^*|e)
\end{split}
\label{eq:pgge}\end{equation}



\begin{figure*}[!hptb]
\label{fig:pgge}
\centering
\begin{tikzpicture}[->,>=stealth',shorten >=1pt,auto,node distance= 3 cm,
  thick,main node/.style={circle,fill=blue!10,draw,font=\sffamily\bfseries,align = center}]

  \node[main node] (4) {piece};
  \node[main node] (2) [left of=4] {painting};

  \node[main node] (5) [below of=2] {oil painting};
  \node[main node] (6) [left of=5] {famous\\painting};
  \node[main node] (7) [left of=6] {world's\\most famous\\painting};

  \node[main node] (10) [right of=5] {art piece};
  \node[main node] (11) [right of=10] {historical\\art piece};

  \node[main node] (12) [below of=6] {Mona lisa};


  \path[every node/.style={font=\sffamily\small}]

    (5) edge  node [left] {$\lambda_{i3}$} (2)
        edge [bend right] node [right] {\small{$ 0.65\alpha_{i3} $}} (2)
        edge  [bend right] node[left]  {\small{$ 0.28\alpha_{i3} $}} (4)
    (6) edge [bend left] node [right] {$\lambda_{i1}$} (2)
    (7) edge [bend left] node [right] {$\lambda_{i2}$} (2)

    (10) edge [bend right] node[right] {$\lambda_{i4}$} (4)
    (11) edge node[right] {$\lambda_{i5}$} (4)
    (12) edge [bend right]node[right] {0.01} (10)
         edge [bend right]node[right] {0.007} (11)
         edge node[right] {0.04} (5)
         edge node[right] {0.05} (6)
         edge node[right] {0.007} (7)
         edge node[left] {0.23} (2)
         edge [bend right]node[right] {0.04} (4);

\end{tikzpicture}
\caption{calculating $P({c_h}|\term{Mona Lisa})$ }
\end{figure*}

The process of calculation is illustrated in the example~\ref{exa:calc}

\begin{example}[Calculating $P({c_h}|e)$]
\label{exa:calc}
As illustrated in Fig.~\ref{fig:pgge}, the process of calculating the typicality a concept is as follows, where \term{painting} is ${c_h}$ and \term{Mona Lisa} is $e$. Then $P(\term{painting}|\term{Mona Lisa})$ consists of 2 parts, the direct edge $P_{original}({c_h}|e)= 0.23$, and the second part
$$\sum_{ c_{l}^*\in C_{l} } [ \lambda_{i}^*+(\alpha_{i}^*) P({c_h}|c_{l}^*) ] \times  P(c_{l}^*|e) $$
$(\alpha_i^*+{c_h}^*=1)$
Thus we get
$$ P = 0.007\times \lambda_{i2}+0.05\times \lambda_{i1}+0.04\times(\lambda_{i3}+0.65\alpha_{i3}) $$
For \term{piece}, it is the similar process. The relation here is only part of the whole graph.
\end{example}




We consider only 2 layers of isA relationship for 2 reasons. The first one is that more layers will lead to noisy concepts such as \term{issue, factor, element}, which are concepts for almost eveything, Secondly, discussing the transitive relation between concepts is beyond the scope of this paper.
